% Giovedì 3 Ottobre 2013

Il rischio è tutto ciò che mina le nostre certezze, ciò che non riusciamo a governare. Bisogna convivere con il rischio e mitigarlo, cercare di prevenire il danno.

\textbf{Risk mitigation}, ma prima bisogna capire quali sono i rischi. Un progetto deve guardare queste cose. Bisogna affrontarli in modo che siano addomesticati.

Un progetto non inizia se non con un'attività importante che faccia emergere i \textbf{requisiti}; la maggior parte dei quali sono inespressi. Gli stakeholder non hanno la capacità o la pazienza di scrivere tutto ciò di cui hanno bisogno. Un fornitore di un prodotto deve \textbf{intuire} queste necessità. Questo si fa con analisi ed emersioni predefinite. Bisogna vedere i requisiti non scritti. Questa attività è un valore aggiunto dominante.

\textbf{Verifica e validazione}, insieme li chiamiamo qualifica ma sono due cose distinte ed occupano un sacco di tempo. Consapevolezza di consegnare al c\textit{ustomer} un prodotto valido. Non dobbiamo far fare a lui il \textit{debug}.

E' irragionevole sviluppare sw da zero (\textbf{from scrutch}). Non è più l'epoca di far le cose così, bisogna puntare sul riuso non sull'"\textit{ex novo}". Il \textit{coding} è un'attività frazionaria con un notevole rischio, quindi meno tempo utilizzo per questo meglio è.

Un processo è un insieme di attività complesse. Le attività sono suddividibili in parti più piccole, chiamate \textbf{task}. Devo \textbf{vivere sulla pianificazione}, suddividere il tempo. Quanto più piccolo è il compito tanto più piccolo è il rischio. Interesse decisivo è spezzare le attività ad una grana fina per ridurre i tempi, costi e rischi. E' importante capire come \textit{frantumare le attività}. Bisogna poi lavorare \textbf{in parallelo}. Dare compiti grandi è fonte di rischi è va evitato. Bisogna concentrarsi su obiettivi raggiungibili facilmente.

C'è un enfasi dominante sulle \textbf{prove} (\textit{tests}). La programmazione è congiunta al \textit{testing}. La misura di produttività in un processo sw è \textit{linee codice/ora}. Il numero massimo di linee produttive è 25 al giorno. Questo sul sw ex-novo. Non è scrivere codice che è importante ma scrivere codice che funzioni alla consegna. Oltre che scrivere codice bisogna anche verificarlo prima di immetterlo nella \textit{repository}.

Si può scendere sotto i tasks, dal basso nascono strumenti e tecniche che dobbiamo acquisire prima di metterci sui tasks. Un compito lo si assumo avendo acquisito consapevolezza degli strumenti e le tecniche per portarlo a termine. Bisogna scegliere uno strumento che sia \textbf{collaborativo}.

Un sw non è fine a se stesso ma è parte di un sistema. Il sistema lo compongono tutti coloro che sono chiamati a usarlo. Bisogna capire bene il sistema. La parte dei requisiti emergerà dal \textbf{dominio}, dall'ambiente. Dicesi organizzazione un aggregato di persone che agiscono secondo regole, sono sistematici, disciplinati e quantificabili. Senza organizzazione c'è il \textbf{chaos}.

Un'impresa certamente funziona a \textbf{pipeline}. L'organizzazione è divisa in \textit{settori}, ciascuno con un proprio compito, che sono finalizzati a un \textbf{flusso}. Ogni settore risponde alle esigenze di attività trasversali. Le regole sono i processi. Prendere gli standard rilevanti ed utilizzare un'istanza. Poi fissare ai processi \textit{specifiche di progetto}. 

La conoscenza su come il sw funzione deve essere scritta. I processi vanno documentati. \textbf{Norme di progetto}, che dicono come un ruolo va svolto.

\textbf{Influenza} la grandezza del progetto, che dice quanto tempo mi ci vorrà. In questo modo potrò fare un conto in base ai tempi e saprò di quante persone avrò bisogno. Dipende anche dalla \textbf{complessità} del progetto e dai \textbf{rischi}, come la competizione, il rischio finanziario... Devo capire la distanza tra le competenze disponibili e quelle richieste. Devo colmare il gap, studiare in modo organizzato.

I processi, che sono l'organizzazione, hanno bisogno di manutenzione, non possono durare per sempre. Voglio migliorare rispetto all'efficienza e all'efficacia. Ciclo di ritorno dall'esperienza, imparare dai propri errori.

I processi produttivi devono avere un \textbf{ciclo interno} atto a migliorarli costantemente. Il ciclo interno di miglioramento è indicato con l'acronimo PCDA (o principio di Deming). Questo ciclo è fatto di 4 attività a ciclo:

\begin{itemize}

	\item \textbf{Plan}, non è il piano delle attività di processo, ma il piano di miglioramento; pianifico ciò che produce efficienza ed efficacia;
	\item \textbf{Do}, stretta attuazione di ciò che ho pianificato. Devo sapere rispetto a cosa migliorare;
	\item \textbf{Check}, guardo l'esito migliorativo delle modifiche fatte rispetto al piano;
	\item \textbf{Act}, ottenuto il risultato porto questo esito a migliorare il processo. Porto miglioramenti o, se fallisco, ragiono su come portare miglioramenti.

\end{itemize}

Il PCDA serve a migliorare l'efficienza e l'efficacia attraverso degli obiettivi di miglioramento. Prima di fare il plan del PCDA devo fare \textbf{analisi}. 

Un ciclo di vita sono gli stati di maturità che ha un sw dalla sua nascita al suo ritiro. Bisogna fornire agli stakeholder evidenza di fattibilità.