% Lunedì 14 Ottobre 2013

\textbf{Ciclo di vita di un software}, conoscere il ciclo di vita ci aiuta a capire la swe. Il sw va inteso in un senso molto più ampio di "\textit{programma}", serve per soddisfare i requisiti di qualcuno. Queste cose evolvono nel tempo e raggiungono \textbf{stati} tramite \textbf{transizioni} scatenate da attività che hanno il fine di far avanzare il sw. Si divide in 4 fasi:

\begin{itemize}

	\item \textbf{Concezione}, quando qualcuno pensa che ci sia (o abbia) bisogno di qualcosa;
	\item \textbf{Sviluppo};
	\item \textbf{Utilizzo};
	\item \textbf{Ritiro}.

\end{itemize}

Questi stati cambiano attraverso attività specifiche, derivanti da processi di cicli di vita.

\textbf{Fase}, è un periodo di tempo contiguo con un inizio ed una fine. Un prodotto sw si spalma sul tempo. Immaginare il ciclo di vita su un asse temporale, le fasi sono segmenti tra uno stato e l'altro.

C'è un processo che si chiama \textbf{documentazione}. Serve a produrre documenti che evidenziano lo stato di maturità di un prodotto sw. Questo processo va utilizzato \textbf{sempre} (tranne che nel ritiro).

E' importante avere un ciclo di vita chiaro in testa, perché averlo ci fa capire quello che dobbiamo fare e se siamo in grado di farlo. Il compito fondamentale è utilizzare un approccio quantificabile, in modo da poter misurare gli obiettivi e le aspettative. Cosa devo fare? Perché? Sono capace in questo modo di puntare a una qualità misurabile.

\textbf{Conformità}, un grado di corrispondenza ad aspettative (efficacia). Misurare quanto soddisfiamo (siamo conformi) le aspettative.

\textbf{Maturità}, è il grado di stabilità con il quale siamo in grado di fare le cose. Siamo maturi quando abbiamo acquisito il "\textit{modus operandi}". Essere sistematici attraverso un processo di maturità.

\textbf{Modello di ciclo}, astrazione del modo nel quale vediamo gli stati e il loro avanzamento.

\textbf{Non modello}, "\textit{code and fix}", raggiungere la correttezza "\textit{by correction}". Non so perché ce l'ho fatta. Prima faccio, poi penso, insieme negativo delle cose da fare. La programmazione è una delle attività meno importanti, deve essere la conseguenza.

I modelli devono essere organizzati. Si identificano 6 modelli organizzativi:

\begin{enumerate}

	\item \textbf{Modello Waterfall}, "\textit{a cascata}". Non è ammesso un ritorno ad una fase già visitata così come non è possibile risalire una cascata. Fatto una volta il codice deve essere apposto. \textit{Document driven}, documenti che spiegano la fiducia nelle scelte. Prima la documentazione poi il sw. Senza i documenti non si va avanti, è una \textit{pre-condizione}. La \textit{post-condizione} è avere un documento approvato. Le fasi sono distinte e non si sovrappongono, tutto è allineato in una linea temporale. Ancora oggi è un modello molto utilizzato. La frustrazione più grande è che il codice arriva tardi. Si programma molto dopo e rispetto alla nostra abitudine è frustrante. Non arrivano inoltre prototipi, bisogna immaginarseli. Questo si fa solo se tutti gli \textit{stakeholder} sono d'accordo sulle pre e le post. E' una cascata senza ritorno. Funziona in un mercato non troppo competitivo. Tipicamente negli appalti pubblici funziona esattamente così;
	
	\item Un \textbf{modello di tipo incrementale} prevede ritorni. E' difficile in questo modello definire una fase. Fisso un quadro generale, inizio a sviluppare la base, poi torno in ciclo e aggiungo delle cose. Ogni realizzazione aggiunge un pezzo. Esempio: sviluppo separato di interfaccia grafica utente e base di dati. Procedendo a cicli di incremento so esattamente quanti cicli farò. In ogni progetto devo dire quando finirò. C'è un servizio di cui si conosce esattamente la durata (\textit{service level agreement});
	
	\item Un \textbf{modello iterativo} farebbe un passo più indietro. Se non ritorno in analisi il problema è completamente definito e non corro il rischio di buttare via cose. Nel modello iterativo torno indietro da realizzazione ad analisi. Si riprova. In questo modello non so dire con certezza quante iterazioni farò. Il modello iterativo lo uso quando gli stakeholder non sanno esattamente cosa vogliono. Si va per tentativi. E' un modello molto rischioso;
	
	\item Nel \textbf{modello evolutivo} si ragiona su un'analisi preliminare (schizzo, un'idea non precisa) poi si inizia a fare una versione. Insegue il futuro non ritirando il passato, ho tante attività concorrenti. Se le cose che voglio fare non le so all'inizio una nuova versione "\textit{asfalta}" molto del passato. Si fa su un prodotto che ha la capacità di assorbire molte versioni sul mercato (esempio i browser). Questo modello lo attua chi può sostenere molte versioni in parallelo (e quindi ha buone capacità finanziarie);
	
	\item \textbf{Modello a spirale}, il problema è il rischio non studiato. Bisogna usare un modello fortemente consapevole dei rischi. A partire dall'inizio ci sono molti cicli ripartiti per capire meglio i rischi e solo dopo la spirale si apre. Per riuscire a trovare una via di uscita si utilizzano i prototipi. Servono a capire se il rischio si può risolvere, confermano se ciò che ho fatto è ragionevole. Questo modello aspira a togliere i rischi; ha avuto molto uso in ambienti \textit{risk driven}, in progetti di carattere "\textit{sistema}";
	
	\item \textbf{Modello a componenti}, nasce sull'osservazione che molto di quello che ci serve esiste già. Pensare che rifare da capo sia molto probabilmente fallimentare. Ragionare per riuso. Componenti riusabili. La progettazione confronta il problema con le realtà esistenti. Adattare i requisiti alle disponibilità. Si negozia con lo \textit{stakeholder}. Adatto i requisiti al panorama delle possibilità e poi progetto riusando le possibilità. La creatività sta nell'integrare e nell'usare bene ciò che esiste già. Richiede però di studiare il dominio.

\end{enumerate}

\textbf{Metodi agili}, opposizione al modello sequenziale. 4 principi su cui si basa:

\begin{enumerate}

	\item Mettere in primo piano persone e iterazioni piuttosto che processi e strumenti. Gli individui sono importanti ed è importante il modo in cui interagiscono tra loro;
	
	\item "\textit{Dei documenti me ne frego, basta che funzioni}". La documentazione non sempre corrisponde a sw funzionante. E' importante che funzioni. Ma nel lungo periodo farà soffrire chi dovrà prendere in mano il sw;
	
	\item Avere un buon rapporto con il customer, coinvolgerlo, non ingessare il rapporto;
	
	\item Reattività piuttosto che pianificazione. Capacità di adattamento a cambiamenti delle situazioni.

\end{enumerate}

Il modello agile è maturato in qualcosa di plausibile, in alcune tecniche che funzionano.

\textbf{Modello agile educato}, tutto ruota intorno al "\textit{document user story}", è l'espressione precisa di quello che l'utente vuole in quel momento. Associato ad ogni user story produrre una strategia che sia in grado di soddisfare quei bisogni. User story associato a un insieme di cose da fare che dimostrino al customer che c'è una soluzione (\textbf{product backlog}). Quello che si ottiene va verso il customer o in feedback. C'è un supervisore che decide l'urgenza strategica (\textbf{sprint}). Tutti fanno un pezzo e poi lo consegnano. Ogni iterazione è rapida e va molto verso le aspettative del cliente.

\textbf{Scrum daily}, mucchio giornaliero, un contatto giornaliero dell'avanzamento. Tecnica ragionevole per chi ha poca esperienza.