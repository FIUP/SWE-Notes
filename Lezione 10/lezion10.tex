%%%%%%%%%%%%%%%%%%%%%%%%%%%%%%%%%%%%%%%%%
% Structured General Purpose Assignment
% LaTeX Template
%
% This template has been downloaded from:
% http://www.latextemplates.com
%
% Original author:
% Ted Pavlic (http://www.tedpavlic.com)
%
% Note:
% The \lipsum[#] commands throughout this template generate dummy text
% to fill the template out. These commands should all be removed when 
% writing assignment content.
%
%%%%%%%%%%%%%%%%%%%%%%%%%%%%%%%%%%%%%%%%%

\documentclass{article}

\usepackage{fancyhdr} % Required for custom headers
\usepackage{lastpage} % Required to determine the last page for the footer
\usepackage{extramarks} % Required for headers and footers
\usepackage{graphicx} % Required to insert images
\usepackage[latin1]{inputenc}

% Margins
\topmargin=-0.45in
\evensidemargin=0in
\oddsidemargin=0in
\textwidth=6.5in
\textheight=9.0in
\headsep=0.25in 

\linespread{1.1} % Line spacing



\setlength\parindent{0pt} % Removes all indentation from paragraphs

%----------------------------------------------------------------------------------------
%	DOCUMENT STRUCTURE COMMANDS
%	Skip this unless you know what you're doing
%----------------------------------------------------------------------------------------

% Header and footer for when a page split occurs within a problem environment
\newcommand{\enterProblemHeader}[1]{
\nobreak\extramarks{#1}{#1 continued on next page\ldots}\nobreak
\nobreak\extramarks{#1 (continued)}{#1 continued on next page\ldots}\nobreak
}

% Header and footer for when a page split occurs between problem environments
\newcommand{\exitProblemHeader}[1]{
\nobreak\extramarks{#1 (continued)}{#1 continued on next page\ldots}\nobreak
\nobreak\extramarks{#1}{}\nobreak
}

\setcounter{secnumdepth}{0} % Removes default section numbers
\newcounter{homeworkProblemCounter} % Creates a counter to keep track of the number of problems
%----------------------------------------------------------------------------------------
%	NAME AND CLASS SECTION
%----------------------------------------------------------------------------------------

\newcommand{\lessonNumber}[1]{Lezione\ \##1} % Assignment title
\newcommand{\lessonDate}[4]{#1,\ #2\ #3\ #4} % Due date
\newcommand{\lessonCourse}[1]{#1} % Course/class
\newcommand{\lessonTime}[1]{#1} % Class/lecture time
\newcommand{\lessonTeacher}[1]{#1} % Teacher/lecturer
\newcommand{\lessonAuthor}[1]{#1} % Your name

% Set up the header and footer
\pagestyle{fancy}
\lhead{\lessonAuthor{Luca De Franceschi}} % Top left header
\chead{\lessonAuthor{Christian Cardin}\ \lessonTime{9:30}: \lessonNumber{10}} % Top center header
\rhead{\firstxmark} % Top right header
\lfoot{\lastxmark} % Bottom left footer
\cfoot{} % Bottom center footer
\rfoot{Page\ \thepage\ of\ \pageref{LastPage}} % Bottom right footer
\renewcommand\headrulewidth{0.4pt} % Size of the header rule
\renewcommand\footrulewidth{0.4pt} % Size of the footer rule

%----------------------------------------------------------------------------------------
%	TITLE PAGE
%----------------------------------------------------------------------------------------

\title{
\vspace{2in}
\textmd{\textbf{\lessonNumber{10}}\\
\normalsize\vspace{0.1in}\small{\lessonDate{Lunedì}{28}{Ottobre}{2013}}\\
\vspace{0.1in}\large{\textit{\lessonTeacher{Christian Cardin},\ \lessonTime{09:30-11:15}}}
\vspace{3in}
}
}

\author{\textbf{\lessonAuthor{Luca De Franceschi}}}
\date{} % Insert date here if you want it to appear below your name

%----------------------------------------------------------------------------------------

\begin{document}

\maketitle
\newpage
\newpage

\textbf{Generalizzazione}, si applica molto bene agli attori, si parla di \textbf{ereditarietà}.

\begin{center}

\includegraphics[width=0.75\columnwidth]{img1} % Example image

\end{center}

Se un attore B estende un attore A l'attore B accede a tutte le funzionalità dell'attore A più le sue caratteristiche. Es. l'amministratore è allo stesso tempo un utente normale ma in più ha altri privilegi. La generalizzazione tra casi d'uso è meno usata.

\begin{center}

\includegraphics[width=0.75\columnwidth]{img2} % Example image

\end{center}

La generalizzazione fra attori è molto più comune e facile da realizzare:

\begin{center}

\includegraphics[width=0.75\columnwidth]{img3} % Example image

\end{center}

L'utente non può essere una generalizzazione di un utente autenticato perchè l'utente autenticato non può effettuare autenticazione.\\
Esempio: l'attore utente può accedere alla funzionalità \textit{autenticazione} del sistema:

\begin{center}

\includegraphics[width=0.75\columnwidth]{img4} % Example image

\end{center}

Scendiamo di dettaglio, meno astratti:

\begin{center}

\includegraphics[width=0.75\columnwidth]{img5} % Example image

\end{center}

\textbf{Diagrammi delle classi e degli oggetti}\\\\

Il paradigma più utilizzato è il paradigma ad oggetti, perchè modella molto bene la realtà. Ho una lista di requisiti, ora il progettista deve produrre i requisiti e descrivere l'architettura del prodotto, e dovrà descriverla in maniera formale, in modo che i programmatori sviluppino esattamente quello che il progettista ha pensato. Posso in questo modo garantire alcune proprietà. Ho bisogno dunque di un linguaggio per parlare ai programmatori. Una classe è una descrizione di qualcosa e l'oggetto è un'istanza che rispetta questa descrizione. Si passa dalla descrizione a qualcosa di tangibile.

\begin{center}

\includegraphics[width=0.75\columnwidth]{img6} % Example image

\end{center}

Modella un concetto ed è indipendente dal linguaggio di programmazione con cui andrò a implementare. La prima cosa che andremo a definire di una classe sono i suoi attributi, che vanno scritti nella parte centrale.

\begin{center}

\texttt{Visibilità nome : tipo [molteplicità] = default [proprietà aggiuntive]}

\end{center}

Questa è la segnatura per la visibilità degli attributi:

\begin{itemize}

	\item \textbf{+}, pubblica;
	\item \textbf{-}, privata;
	\item \textbf{\#}, protetta.

\end{itemize}

\begin{center}

\includegraphics[width=0.75\columnwidth]{img7} % Example image

\end{center}

La molteplicità è 1 perchè una persona ha solo un'età. L'attributo può essere anche espresso come \textbf{associazione} tra due tipi. Questo si fa con una freccia orientata dalla classe che contiene una copia dell'altro tipo. Associazioni senza verso sono bidirezionali. Si utilizzano gli attributi testuali per i tipi primitivi (nella libreria del linguaggio che stiamo utilizzando), mentre si utilizzano le associazioni quando ci si riferisce a due classi del nostro dominio. Se abbiamo molteplicità superiore a 1 significa che abbiamo una collezione (array, liste, ...). Possono esserci delle convenzioni per gli attributi (es. definizione obbligatoria dei metodi \textit{setter} e \textit{getter}. Le proprietà sono gli attributi e le associazioni.\\
Le operazioni sono ciò che la classe espone verso l'esterno, sono i "servizi" della classe.

\begin{center}

\texttt{Visibilità nome (lista-parametri} : tipo-ritorno {proprietà aggiuntive\}}\\
\texttt{Lista-proprietà := direzione nome : tipo = default}

\end{center}

\begin{center}

\includegraphics[width=0.75\columnwidth]{img8} % Example image

\end{center}

Le \textbf{query} sono tutte le operazioni che non modificano l'oggetto di invocazione, a differenza dei metodi modificatori. Operazione != metodo, concetto differente in presenza di polimorfismo.\\\textbf{Commenti e note}

\begin{center}

\includegraphics[width=0.75\columnwidth]{img9} % Example image

\end{center}

Un concetto fondamentale è la dipendenza fra due tipi. Una classe A dipende da B se una modifica fatta a B implica una modifica ad A. Le dipendenze sono il "male assoluto" e vanno minimizzate, perchè le classi devono essere autoconsistenti. Più dipendenze ho e più una modifica può creare \textit{side-effect} su un'altra classe (problemi in fase di manutenzione). Un modo per minimizzare le dipendenze è l'uso di interfacce.\\
Le dipendenze in UML sono di vario tipo, c'è bisogno di un classificatore, al fine di comprenderla meglio. Questo si inserisce come etichetta nella freccia.

\begin{center}

\includegraphics[width=0.75\columnwidth]{img10} % Example image

\end{center}

L'aggregazione e la composizione sono particolari tipi di associazione. L'aggregazione si identifica con la frase "\textit{parte di...}", gli aggregati possono essere condivisi. Viene rappresentato con un diamante vuoto. La composizione è come l'aggregazione ma le istanze i un'aggregazione possono appartenere solo ad un aggregato. Solo l'oggetto intero può creare e distruggere le sue parti.

\begin{center}

\includegraphics[width=0.75\columnwidth]{img11} % Example image

\end{center}

\begin{center}

\includegraphics[width=0.75\columnwidth]{img12} % Example image

\end{center}

\begin{center}

\includegraphics[width=0.75\columnwidth]{img13} % Example image

\end{center}

Può succedere che un'associazione abbia bisogno di essere specificato maggiormente. In questo caso si creano \textbf{classi di associazioni}, che aggiungano attributi ed operazioni alle associazioni. Ma i linguaggi i programmazione non prendono un'implementazione di queste cose.

\begin{center}

\includegraphics[width=0.75\columnwidth]{img14} % Example image

\end{center}

La generalizzazione è un concetto molto importante perchè descrive l'ereditarietà, uno dei concetti fondamentali della programmazione a oggetti. A generalizza B se ogni oggetto di B è anche un oggetto di A. Sottotipo != Sottoclasse.

\begin{center}

\includegraphics[width=0.75\columnwidth]{img15} % Example image

\end{center}

Per le classi astratte si usa il nome in \textit{corsivo}, non può essere instanziata perché ha delle operazioni che non possiedono l'implementazione, anche se ne può possedere alcune implementate.\\
Un altro concetto è quello di \textbf{interfaccia}, che non è una classe (al massimo è un tipo) ed è priva di implementazione. Il loro scopo è definire un contratto che le classi che la implementeranno devono assolutamente fornire.

\begin{center}

\includegraphics[width=0.75\columnwidth]{img16} % Example image

\end{center}

\end{document}