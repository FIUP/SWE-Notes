%%%%%%%%%%%%%%%%%%%%%%%%%%%%%%%%%%%%%%%%%
% Structured General Purpose Assignment
% LaTeX Template
%
% This template has been downloaded from:
% http://www.latextemplates.com
%
% Original author:
% Ted Pavlic (http://www.tedpavlic.com)
%
% Note:
% The \lipsum[#] commands throughout this template generate dummy text
% to fill the template out. These commands should all be removed when 
% writing assignment content.
%
%%%%%%%%%%%%%%%%%%%%%%%%%%%%%%%%%%%%%%%%%

\documentclass{article}

\usepackage{fancyhdr} % Required for custom headers
\usepackage{lastpage} % Required to determine the last page for the footer
\usepackage{extramarks} % Required for headers and footers
\usepackage{graphicx} % Required to insert images
\usepackage[latin1]{inputenc}

% Margins
\topmargin=-0.45in
\evensidemargin=0in
\oddsidemargin=0in
\textwidth=6.5in
\textheight=9.0in
\headsep=0.25in 

\linespread{1.1} % Line spacing



\setlength\parindent{0pt} % Removes all indentation from paragraphs

%----------------------------------------------------------------------------------------
%	DOCUMENT STRUCTURE COMMANDS
%	Skip this unless you know what you're doing
%----------------------------------------------------------------------------------------

% Header and footer for when a page split occurs within a problem environment
\newcommand{\enterProblemHeader}[1]{
\nobreak\extramarks{#1}{#1 continued on next page\ldots}\nobreak
\nobreak\extramarks{#1 (continued)}{#1 continued on next page\ldots}\nobreak
}

% Header and footer for when a page split occurs between problem environments
\newcommand{\exitProblemHeader}[1]{
\nobreak\extramarks{#1 (continued)}{#1 continued on next page\ldots}\nobreak
\nobreak\extramarks{#1}{}\nobreak
}

\setcounter{secnumdepth}{0} % Removes default section numbers
\newcounter{homeworkProblemCounter} % Creates a counter to keep track of the number of problems
%----------------------------------------------------------------------------------------
%	NAME AND CLASS SECTION
%----------------------------------------------------------------------------------------

\newcommand{\lessonNumber}[1]{Lezione\ \##1} % Assignment title
\newcommand{\lessonDate}[4]{#1,\ #2\ #3\ #4} % Due date
\newcommand{\lessonCourse}[1]{#1} % Course/class
\newcommand{\lessonTime}[1]{#1} % Class/lecture time
\newcommand{\lessonTeacher}[1]{#1} % Teacher/lecturer
\newcommand{\lessonAuthor}[1]{#1} % Your name
\begin{document}

\section{Lezione 6}

\textbf{Gestione qualità}: la possiamo guardare da due punti di vista: qualità sul prodotto e qualità su ciò che si fa (qualità di processo). È una competenza strategica, una funzione aziendale e non un ruolo di progetto. Serve come indicatore che ``stiamo lavorando bene''. Dobbiamo lavorare prima per fare meno manutenzione poi.

\textbf{Pianificazione di progetto}: serve a sapere cosa dobbiamo fare nell'assicurarci che le nostre attività producano esiti che aiutino la nostra collaborazione; regolare l'avanzamento. Cominciamo ad identificare ``cosa c'è da fare'', pianifichiamo le attività, flussi di attività e flussi di azione. Chi lo può fare? Quanto mi costa? Le attività hanno un flusso, hanno un inizio ed un'uscita. Si dipanano su un asse temporale. Capite le attività mi devo chiedere chi posso mettere lì, come ruoli (non come persone). Attività che hanno dei vincoli sugli estremi ed eventualmente in mezzo. Devo capire quanto mi servirà per svolgere un'attività data, tempo, persone, denaro. Le ore vanno intese come ore di calendario: tempo/persona, quanto tempo uso tali persone sapendo che ciascuna percepisce un costo orario preciso. In base a questo posso capire quanto mi costerà un progetto.

\textbf{Diagrammi di Gantt}: strumento intellettuale semplice per mettere in relazione varie cose. Prima cosa da fare: determino le attività da svolgere su tutta la durata o su periodi più piccoli. L'orizzonte è dato in parte dal modello di ciclo di vita che adotto. Poi determinare le dipendenze, che consumano tempo e non producono. Terza cosa, dire quanto tempo e persone mi serviranno per una certa attività; una volta decise le risorse, alloco persone alle attività. Poi mettiamo a diagramma le cose e vediamo se tornano, altrimenti si torna indietro (ciclo). I diagrammi mi aiutano a capire se riesco a gestire il tutto. I diagrammi di Gantt sono stati creati oltre 100 anni fa; sull'asse orizzontale c'è il tempo segmentato per unità di lavoro (es. 1gg, una settimana, un mese..). Una buona scelta è utilizzare come unità di misura un giorno (8 ore lavorative); ma ho un calendario (es. sabato e domenica non si lavora). Sull'asse orizzontale ci sono tutte le attività che devo svolgere e archi che rappresentano le transizioni tra una e l'altra. Le dipendenze logiche tra attività: quelle che non hanno dipendenze partono subito, la seconda è la durata minima di tempo/persone che posso usare per una determinata attività. \textit{Work allocation}, in questa fase non sto 
allocando persone ma ruoli.

\textbf{PERT}: (Program Evaluation and Review Technique) che serve a calcolare il rischio di calendario che consiste nell'avere un realistico margine tra un input e un output (slack). Se misuro lo slack e mi accorgo che è 0 o negativo ho una criticità. È bene sapere che non ci sia questo rischio. Non si può avere nemmeno slack troppo grande. Diagrammi specifici che collocano gli slack e rappresentano il rischio. Indica l'ordine delle attività e la loro durata e indica la prima data in cui posso iniziare un'attività che abbia tutti gli input necessari. Calcolando il cammino più lungo tra l'inizio e la fine del progetto posso calcolare l'attività che finisce prima. Questo diagramma legge il Gantt rispetto allo slack e calcola i cammini critici. 

Gantt e PERT sono due diagrammi potentissimi che possiamo creare via software, tramite strumenti specifici. In generale sono dotazione di un buon ambiente di programmazione. Sono strumenti essenziali fin dall'inizio.

Altro tipo di diagramma è il \textbf{Work Breakdown Structure}, strumento per capire quali sono le attività utili/fondamentali tramite decomposizione (struttura di decomposizione del lavoro). L'attività viene scomposta e numerata con numerazione gerarchica. Ogni componente deve essere univocamente identificabile. Un'attività decomposta permette di agevolare il parallelismo. Se non frammento ammasso un grosso rischio su una persona. Attenzione moto acuta sulla dimensione di attività. Il WBS serve ad avere una percezione delle attività fini (foglie) da svolgere. Allocazione delle risorse: assegnare attività a ruoli e poi ruoli a persone per far si che ogni persona sia impegnata per tutto il tempo disponibile. Diagramma di Gantt ribaltato sulle persone (al posto delle attività); l'intento è sapere che sto usando le persone in parallelo ed entro il limite dato. La decisione delle ore di investimento per la formazione personale è a discrezione di ciascuno. La componente informatica moderna è basata sull'apprendimento di tecnologie nuove.

Fattori di influenza:

\begin{itemize}

\item \textbf{Dimensione del progetto}: quante righe di codice mi servono; nel nostro progetto starà tra 5000-10000 righe, nello stage circa 2000 (si intende righe produttive). La dimensione si riesce a stimare macroscopicamente; sta tra due assi: quanto codice e quanto complesso;

\item \textbf{Esperienza del dominio}: quanta è l'esperienza di chi svolge il progetto nei confronti del dominio a cui siamo davanti;

\item \textbf{Tecnologie adottate}: influenza molto forte, è una scelta strategica di amministrazione;

\item \textbf{Ambiente di sviluppo}: procedure e tecniche con le quali collaboriamo, dobbiamo collocale a basso costo temporale;

\item \textbf{Qualità di richiesta}.

\end{itemize}

\textbf{Problematiche di stima}

\textbf{Legge di Parkinson}: il lavoro si espande fino a riempire il tempo disponibile.

\textbf{Legge della domanda}: quanto più è basso il prezzo di una cosa tanto più grande sarà la domanda per quella cosa lì. Uso in abbondanza le cose che costano poco.

\end{document}